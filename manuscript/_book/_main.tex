% Options for packages loaded elsewhere
\PassOptionsToPackage{unicode}{hyperref}
\PassOptionsToPackage{hyphens}{url}
%
\documentclass[
  12pt,
]{article}
\usepackage{amsmath,amssymb}
\usepackage{lmodern}
\usepackage{iftex}
\ifPDFTeX
  \usepackage[T1]{fontenc}
  \usepackage[utf8]{inputenc}
  \usepackage{textcomp} % provide euro and other symbols
\else % if luatex or xetex
  \usepackage{unicode-math}
  \defaultfontfeatures{Scale=MatchLowercase}
  \defaultfontfeatures[\rmfamily]{Ligatures=TeX,Scale=1}
\fi
% Use upquote if available, for straight quotes in verbatim environments
\IfFileExists{upquote.sty}{\usepackage{upquote}}{}
\IfFileExists{microtype.sty}{% use microtype if available
  \usepackage[]{microtype}
  \UseMicrotypeSet[protrusion]{basicmath} % disable protrusion for tt fonts
}{}
\makeatletter
\@ifundefined{KOMAClassName}{% if non-KOMA class
  \IfFileExists{parskip.sty}{%
    \usepackage{parskip}
  }{% else
    \setlength{\parindent}{0pt}
    \setlength{\parskip}{6pt plus 2pt minus 1pt}}
}{% if KOMA class
  \KOMAoptions{parskip=half}}
\makeatother
\usepackage{xcolor}
\IfFileExists{xurl.sty}{\usepackage{xurl}}{} % add URL line breaks if available
\IfFileExists{bookmark.sty}{\usepackage{bookmark}}{\usepackage{hyperref}}
\hypersetup{
  pdftitle={Voxel-wise Intermodal Coupling Analysis of Two or More Modalities using Local Covariance Decomposition},
  hidelinks,
  pdfcreator={LaTeX via pandoc}}
\urlstyle{same} % disable monospaced font for URLs
\usepackage[margin=1in]{geometry}
\usepackage{longtable,booktabs,array}
\usepackage{calc} % for calculating minipage widths
% Correct order of tables after \paragraph or \subparagraph
\usepackage{etoolbox}
\makeatletter
\patchcmd\longtable{\par}{\if@noskipsec\mbox{}\fi\par}{}{}
\makeatother
% Allow footnotes in longtable head/foot
\IfFileExists{footnotehyper.sty}{\usepackage{footnotehyper}}{\usepackage{footnote}}
\makesavenoteenv{longtable}
\usepackage{graphicx}
\makeatletter
\def\maxwidth{\ifdim\Gin@nat@width>\linewidth\linewidth\else\Gin@nat@width\fi}
\def\maxheight{\ifdim\Gin@nat@height>\textheight\textheight\else\Gin@nat@height\fi}
\makeatother
% Scale images if necessary, so that they will not overflow the page
% margins by default, and it is still possible to overwrite the defaults
% using explicit options in \includegraphics[width, height, ...]{}
\setkeys{Gin}{width=\maxwidth,height=\maxheight,keepaspectratio}
% Set default figure placement to htbp
\makeatletter
\def\fps@figure{htbp}
\makeatother
\setlength{\emergencystretch}{3em} % prevent overfull lines
\providecommand{\tightlist}{%
  \setlength{\itemsep}{0pt}\setlength{\parskip}{0pt}}
\setcounter{secnumdepth}{5}
\usepackage{booktabs}
\ifLuaTeX
  \usepackage{selnolig}  % disable illegal ligatures
\fi
\usepackage[]{natbib}
\bibliographystyle{plainnat}

\title{Voxel-wise Intermodal Coupling Analysis of Two or More Modalities using Local Covariance Decomposition}
\author{true \and true \and true \and true}
\date{02 December, 2021}

\begin{document}
\maketitle

{
\setcounter{tocdepth}{2}
\tableofcontents
}
\hypertarget{credit-author-statement}{%
\section{CRediT author statement}\label{credit-author-statement}}

Fengling Hu: Conceptualization, Methodology, Software, Validation, Formal analysis, Investigation, Writing - Original Draft, Writing - Review \& Editing, Visualization

\url{https://www.elsevier.com/authors/policies-and-guidelines/credit-author-statement}

``Fengling Hu\^{}2, Sarah M. Weinstein, Erica B. Baller, Alessandra M. Valcarcel, Azeez Adebimpe, Armin Raznahan, David R. Roalf, Tim Robert-Fitzgerald, Virgilio Gonzenbach, Ruben C. Gur, Raquel E. Gur, Simon Vandekar, John A. Detre, Kristin A. Linn, Aaron Alexander-Bloch, Theodore D. Satterthwaite* , Russell T. Shinohara *''

\hypertarget{abstract}{%
\section{Abstract}\label{abstract}}

When individual subjects undergo imaging with multiple modalities, biological data is present not only within each modality, but also between modalities - that is, in how modalities covary at the voxel level. Previous studies have shown that the covariance structures between modalities, or intermodal coupling (IMCo), can be estimated between two modalities, and that two-modality IMCo reveals otherwise undiscovered patterns in neurodevelopment as well as other processes. However, previous IMCo methods are based on the slopes of local weighted linear regression lines, which are inherently asymmetric and limited to the two-modality setting. Here, we present a PCA-based generalization of IMCo which uses local covariance decompositions to define a symmetric, voxel-wise coupling coefficient valid for two or more modalities. We then demonstrate this method is spatially heterogeneous and varies with respect to age and sex over the course of neurodevelopment. As availability of multi-modal data increases, PCA-based IMCo offers a natural approach for summarizing relationships between multiple aspects of brain structure and function. An R package is provided.

\hypertarget{introduction}{%
\section{Introduction}\label{introduction}}

Hello test!

\hypertarget{methods}{%
\section{Methods}\label{methods}}

\hypertarget{results}{%
\section{Results}\label{results}}

\hypertarget{discussion}{%
\section{Discussion}\label{discussion}}

\hypertarget{supplementary-materials}{%
\section{Supplementary Materials}\label{supplementary-materials}}

  \bibliography{book.bib,packages.bib}

\end{document}
